\chapter{Existing Methods for Automated Isotope Identification}

\section{What Is and Is Not Possible with Improved Energy Resolution}
\subsection{Regions of Significant Peak Overlap}
\subsubsection{Pu-239}
\subsubsection{Methods to Resolve this Problem}

\section{Setting an Upper Energy Cutoff}
\subsection{How Detectors Limit This}
\subsection{Benefit of Neutron Capture Lines at Higher Energies}

\section{Naturally-Occuring Radioactive Material (NORM)}
\subsection{What is Background?}
\subsection{The Statistics of Background Subtraction}
\subsection{To Subtract or Not to Subtract?}
\subsubsection{The Problems Caused by Technically-Enhanced NORM}

\section{Peak-Based Identification Methods}
\subsection{Expert Systems}
\subsection{Region-of-Interest (ROI) Methods}
\subsection{First Derivative Methods}
\subsection{Second Derivative Methods}
\subsubsection{Mariscotti's Technique}
\subsection{Wavelets and Other Signal Processing-Based Approaches}
\subsection{Benefits and Drawbacks of Peak-Based Methods}

\section{Methods Using the Complete Spectrum}
\subsection{Priciple Component Analysis}
\subsection{Template Matching}
\subsection{Machine Learning Approaches}
\subsection{Benefits and Drawbacks of Analyzing the Complete Spectrum}
