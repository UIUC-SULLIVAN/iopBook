\chapter{Radiation Types and Interactions}

\section{Types of Radiation}

The measurement of radioactive isotopes occurs through the detection and quantification of ionizing radiation.  There are several types of radiation that can ionize matter ranging from neutral particles to waves to charged particles.  Ionizing radiation is specifically a particle or wave that has enough energy to ionize an atom that it might interact with.  As such, the energies that we shall be concerned with tend to correspond to at least the binding energy of an atom, which is on the order of a few electron volts (eV).  Measuring a single ionization is incredibly difficult though, so in reality we will discuss incident radiation whose energy is high enough to cause several ionizations -- typically on the order of 1 keV and higher.

In this chapter we shall study the four main types of ionizing radiation --  alpha particles, beta particles, high-energy photons, and neutrons -- and their common interactions in matter.  There are, of course, a host of other types of radiation that we could attempt to detect such as neutrinos and fission fragments.  While important, it will become clear in this chapter that the four mentioned particles are responsible for the majority of information required for isotope identification.  

In fact, isotope identification can almost always be achieved through the analysis of the energy distribution from the photons (gamma rays and x-rays) emitted by the isotope.  This energy distribution is recorded as a histogram called a spectrum and is the focus of the majority of this text.  That being said, alpha particles, beta particles, and neutrons all are capable of being either directly or indirectly in a spectrum and thus it is important to study them as well.

\subsection{Alpha particles}
\subsection{Beta particles}
\subsection{Photons}
\subsection{Neutrons}

\section{Photon Interactions}
As previously stated, photons such as gamma rays and x-rays are capable of providing a unique finger print for the source isotope.  Both behave similarly when they interact with matter since after creation they are just a photon.  Their key difference is in how they are created.  Gamma rays are emitted from the decay of an unstable nucleus whereas x-rays are created when orbital electrons are rearranged.  Without additional energy being provided in either case, the preference in nature is to decay to a lower energy state.  In the case of gamma rays, energy is released in the form of a photon as the nucleus de-excites.  X-rays, on the other hand, are released as the orbital electrons de-excite to lower energy shells.  In either case, the energy differences between the excited state and de-excited state are quantized.    Therefore, photons emitted from a decaying isotope are emitted at discrete energies that correspond to the different energy states of the atom in question.  An example of these types of decay in shown in Figure ***.  

\missingfigure[figwidth=6cm]{Comparison of the decay schemes of an x-ray emitter (left) and a gamma-ray emitter (right).}

It is because of this fact that high-energy photons can provide a unique identification of the isotope in question.  If the energy of the photon or photons emitted could be measured with infinite precision then the task of identifying the responsible isotope would be trivial.  Unfortunately, it is possible for the photon to interact with matter in a variety of ways prior to detection or within the detector itself that result in the loss of energy prior to complete detection.  Therefore, it is important to consider the interaction mechanisms for high-energy photons and how they result in differences in the energy distribution of the photons away from the discrete intial value.

\subsection{Photoelectric Effect}
As implied by the name, the photoelectric effect includes both photons and electrons.  In this process, the photon is absorbed by an atom and electron in released in its place.  The photoelectric effect was first observed by Heinrich Hertz when he noted an increase in the probability of creating electric sparks when a conductor was illuminated with ultraviolet light. \cite{hertz1887} Planck, Einstein, and Millikan, to name a few, also contributed through theoretical and experiemental study to our understanding of the photoelectric effect. \cite{einstein1905,millikan1914,millikan1916}



\subsection{Compton Scatter}
\subsection{Pair Production}
\subsection{Photofission}

\section{Neutron Interactions}
\subsection{Scattering Events}
\subsubsection{Elastic Scattering}
\subsubsection{Inelastic Scattering}
\subsection{Absorptive Events}
\subsubsection{Charged}
\subsubsection{Electromagnetic}
\subsubsection{Neutral}
\subsubsection{Fission}

\section{Beta Particle Interactions}
\subsection{Brehmsstrahlung}
\subsection{Annihilation}
\subsubsection{Pair Production}



\bibliographystyle{plain}
\bibliography{references}


