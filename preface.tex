\chapter{Preface}

Since the discovery of ionizing radiation by Marie Curie there has been the need to measure and quantify this thing that cannot be detected by the human senses.  The advent of the nuclear age with the nuclear weapon developed by the Manhattan Project at Los Alamos reinforced the need for ways to find radioactive sources that could originate from special nuclear materials.  But perhaps the most defining moment for the importance of radiation detection and quantification came on September 11, 2001.  I was in the final year of my doctoral program in nuclear engineering at the University of Michigan, working in my lab on a new type of radiation detector when the towers came down.  At the time, my lab was in the Phoenix Memorial Laboratory which was attached to the Ford Nuclear Reactor.  The building was immediately locked down and we were not permitted to leave for a few hours.  When I was able to leave the building, I wandered over to the office of my thesis advisor, Prof. Zhong He.  The dazed look on my face was mirrored by everyone I encountered on that walk.  I entered Zhong's office and sat down.  Neither of us spoke for a few minutes and just sat there in a stunned silence.  Zhong, as I had learned, had more brushes with death of anyone I had ever met including leaving Tiennamen Square minutes before the Chinese army started shooting people in the famous protests of the late 1980's.  He spoke first and his words were something I would never forgot.  He said to me, ``I have always written proposals to our sponsors for research funding saying that our detectors could help in the fight against nuclear terrorism, but until today I didn't really know what that meant."

It was because of 9/11 that I, like many other newly-minted PhD's, decided to take a job at Los Alamos National Laboratory (LANL) so I could put my skills to use in fighting this thing called nuclear terrorism.  The Department of Homeland Security had not yet been fully stood up and everywhere within the U.S. goverment different agencies were working around the clock attempting to secure the country from a new attack, whether it be from conventional weaponry or through weapons of mass destruction (WMD).  In my corner of the laboratory a great deal of effort was spent on getting radiation detectors into the hands of the so-called ``first responders" -- those people like police officers, fire fighters, and Customs inspectors who would most likely be the first people to come across a potential terrorist nuclear device.  The first responders were spending billions of dollars purchasing detectors that would be able to not only detect radiation, but also to identify the isotope or isotopes that generated that signature through their gamma-ray spectra.

As we gained more experience with the commerically-available radioisotope identifiers (RIIDs) devices that the first responders were buying, we started realizing that they were rather deficient in one key ability -- the ability to identify isotopes.  The RIIDs of the day typically used sodium iodide scintillators, which are an inexpensive and low-resolution spectrometer.  As will become evident further into this text, the problem of identifying isotopes through spectroscopic analysis can be harder as the resolution degrades.  However, this was not the case with these RIIDs.  My colleagues and I, trained spectroscopists, were able to identify the isotopes within the spectra within seconds to minutes.  So it was the automated algorithm that was not working.

The purpose of this text is to present the reader with the tools necessary to analyze the gamma-ray spectrum from any type of spectrometer and identify and quantify the isotope or isotopes present.  The reader is assumed to be a senior-level undergraduate student or graduate student with a background in  physics or nuclear engineering.  This work will present material on radiation types and interactions with matter at an introductory level with the intent of developing how these different types and interactions are evident in the overall spectrum.  

In so much as possible, the spectra presented in this book will be real, measured spectra as opposed to simulated spectra.  While there are a great many high-quality simulation tools available to generate fake spectra, no simulation is capable of completely representing a true measurement environment.  In such environments, the temperature can drift, a host of materials (naturally radioactive or otherwise) may be present that can change the scatter profile in the spectrum, people can walk between the source and the detector.  There are an infinite number of variables that can perturb the spectrum and it is impossible to simulate all of them.  Therefore, we have prioritized the presentation of actual, measured spectra.  That being said, it is not always practical or possible to measure spectra in every scenario described in this book.  In these cases, it is often necessary to resort to a simulation.  When this is done, it is noted in the corresponding figure.

This book would not be possible without the assistance and support of many individuals.  In no particular order, this includes many of my colleagues at LANL: Paul Felsher, David Mercer, Steven Myers, Brian Rooney.  Several of my students from the University of Illinois have also contributed to this text, including Mark Kamuda, Jacob Stinnett, and Sheng Yang.  Most importantly, I would like to thank my husband, Michael, and daughter, Sophie.  They are truly the lights of my life.